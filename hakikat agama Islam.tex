\documentclass[]{article}

%opening
\author{Mohammad Fajar}



\hyphenation{a-ki-bat-nya}
\hyphenation{ke-ting-gi-an}
\hyphenation{pe-nyim-pa-ngan}
\hyphenation{mo-le-kul}

\hyphenation{ber-da-sar-kan} 



\usepackage{graphicx}


\usepackage{setspace}
\usepackage{float}
\usepackage{caption}
\usepackage{indentfirst} 
\usepackage{amsmath}
\usepackage{setspace}

\usepackage[utf8]{inputenc}
%\usepackage{dirtytalk} % for quotation
%\usepackage{subfigure}

\usepackage{csquotes}

\usepackage{float}
\usepackage{caption}
\usepackage{subcaption}

\usepackage{mathptmx}  %for times news roman font

\usepackage[hyphenbreaks]{breakurl} %break url
\usepackage[hyphens]{url} % ingat paket hyperref itu diletakkan sesudah paket ini biar tidak konflik. 
\usepackage{hyperref}


\usepackage[top=3cm, bottom=3cm, left=3cm, right=3cm]{geometry}


\setlength{\parindent}{.7cm}
% \setstretch{1.25}

%\onehalfspacing


\setlength\parskip{10pt}


\usepackage{enumitem}

%\usepackage{enumitem}
\setlist{parsep=0pt,listparindent=\parindent}
\newlist{mylist}{enumerate}{5}
\setlist[mylist]{label=Caso \arabic*,font=\bfseries}


\newcommand{\satuanmassajenis}{ \textrm{kg}/\textrm{m}^3}
\renewcommand{\refname}{Referensi}
\newcommand{\satuankonstantagravitasi}{\textrm{m}^{-3} \textrm{kg}^{-1} \textrm{s}^{-2}}
\newcommand{\konstantagravitasi}{6.67384 \times 10^{-11} \satuankonstantagravitasi} 
\newcommand{\massabumi}{5.9722 \times 10^{24} \textrm{ kg}}

\newcommand{\jarijaribumi}{6371 \times 10^{3} \textrm{ m}}

\renewcommand{\figurename}{Gambar}

%\numberwithin{equation}{section} 



\hyphenation{me-nga-bur-kan}

\usepackage{titlesec}
%\renewcommand{\thesection}{\thesection\arabic{section}.}
%\usepackage{secdot}
% \usepackage{titlesec}
\titlelabel{\thetitle.\quad}





\begin{document}
Keberadaan dan Keesaan Allah
Keberadaan Allah memiliki banyak bukti yang sangat sulit untuk dibantah. Beberapa ulama yang mulia bahkan mengatakan bahwa Allah adalah zat yang paling nyata, akan tetapi mereka yang tidak memiliki pemahaman tidak akan bisa melihat-Nya. Yang lainnya berkata perwujudan-Nya begitu nyata sehingga menutupi-Nya dari pengamatan secara langsung. 
Akan tetapi, pengaruh yang besar dari positifisme dan materialisme terhadap sains dan umat manusia pada masa sekarang ini membuat sebuah keharusan untuk membahas argumen semacam itu. Karena adalah hal yang umum dalam pandangan ilmiah untuk menyederhanakan keberadaan (eksistensi) sesuatu pada apa yang dirasakan secara langsung, ini akan membutakan kita terhadap suatu dimensi keberadaan yang begitu luas nan jauh. Untuk menghilangkan tabir yang tercipta, kita akan meninjau secara singkat beberapa bukti tradisional mengenai keberadaan Allah. 
Sebelum memulainya, mari kita bercermin pada sebuah fakta sejarah yang sederhana: sejak permulaan kehidupan manusia, teramat banyak dan sifatnya mayoritas manusia yang percaya tentang keberadaan Allah.  Kepercayaan ini saja sudah cukup untuk menetapkan keberadaan Allah itu sendiri.  Orang-orang yang tidak beriman tidak bisa mengatakan bahwa mereka  lebih pandai ketimbang orang-orang yang beriman. Beberapa dari ilmuwan yang paling kreatif, juga para pakar, peneliti adalah orang-orang yang beriman, seperti halnya pakar di bidangnya: para nabi dan wali-wali. 
Ditambah lagi, orang-orang biasanya kebingungan membedakan antara penolakan kita terhadap keberadaan sesuatu dengan penerimaan kita akan ketidakberadaan sesuatu. Jika yang pertama sekedar mengingkari, maka yang kedua adalah memastikan sesuatu dan ini membutuhkan bukti. Tidak satupun yang pernah membuktikan ketidakberadaan Allah, karena untuk melakukan hal tersebut adalah hal yang mustahil, di mana ada tidak terhitung banyaknya dalil yang membuktikan keberadaannya.  Hal ini dapat diperjelas dengan menggunakan perumpamaan berikut. 
Bayangkan sebuah tempat dengan seribu pintu untuk masuk ke dalamnya, di mana 999 dari pintu tersebut adalah terbuka dan satu di antaranya kelihatan seperti tertutup. Jika diberikan hal ini, maka sangat tidak beralasan untuk mengatakan bahwa tempat tersebut tidak bisa dimasuki. Orang-orang yang tidak beriman (kafir atau mungkin ateis) adalah seperti seseorang, yang ingin memastikan bahwa tempat tersebut tidak bisa dimasuki, menutup perhatiannya (juga orang lain) hanya pada pintu yang kelihatan tertutup tersebut. 
Dalil Tradisional Bagi Keberadaan Allah
1. Segala hal sifatnya tidak pasti, karena setara baginya untuk bisa ada atau tidak ada. Segala hal bisa saja ada pada waktu tertentu dan di mana saja, dalam bentuk apapun, dan dengan sifat-sifat apapun. Tidak ada satupun atau tidak seorang pun yang memiliki peran dalam menentukan dalam cara bagaimana, pada waktu kapan, serta tempatnya, bagi sesuatu untuk bisa ada, atau sifat-sifat dan ciri-cirinya. Jadi, harus ada sebuah kekuatan yang memilih antara adanya dia atau tidak adanya dia, dan kekuatan itu harus memberinya sifat-sifat yang unik. Kekuatan ini haruslah tidak berhingga, memiliki keinginan yang mutlak, dan memiliki pengetahuan yang mencakup segala hal. Sudah pasti, itu adalah Allah.
2. Segala sesuatu berubah. Dengan demikian ia berada di ruang dan waktu tertentu, yang artinya ia memiliki awal dan juga memiliki akhir. Suatu hal yang berawal, membutuhkan sesuatu yang tidak berawal  untuk menjadikannya ada, karena ia tidak mungkin berawal dengan sendirinya, sebab jika demikian ini akan menimbulkan pergeseran siapa yang lebih awal hingga tak berhingga. Karena akal tidak mungkin menerima situasi tersebut, sesuatu yang paling awal yang hadir dengan sendirinya, cukup dengan sendirinya, dan kebal terhadap perubahan itu lah yang dibutuhkan.  Dia yang paling awa inilah disebut sebagai Allah.
3. Hidup penuh dengan misteri (ilmuwan tidak bisa menjelaskannya dengan sebab-sebab kebendaan atau menemukan asal mulanya) dan transparan (dalam artian menampilkan daya kreasi). Dengan hal ini, maka kehidupan akan mengatakan: �Allah lah yang menciptakanku.�
4. Segala sesuatu yang ada, dan alam semesta keseluruhannya, memperlihatkan harmoni dan keteraturan padanya dan dalam hubungannya dengan yang lainnya. Keberadaan satu bagian menghendaki keberadaan yang lain secara keseluruhan,  dan keberadaannya secara keseluruhan menghendaki keberadaan bagian-bagiannya agar bisa ada. Sebagai contoh, sebuah sel yang rusak bisa saja akan merusakkan seluruh tubuh. Demikian pula, sebuah pohon membutuhkan kerja sama dan saling menopang dari keberadaan udara, air, tanah, juga saling kerja sama antara satu dengan lainnya untuk bisa ada. Keharmonisan dan kekompakan ini akan menunjuk pada keteraturan dari pencipta, yang mengetahui hubungan dan karakteristik dari segala hal, dan bisa memerintah segala hal. Sang Pencipta ini adalah Allah.
5. Segala ciptaan memperlihatkan limpahan karya seni yang menyilaukan. Namun itu bisa dihadirkan, seperti yang kita lihat, dengan begitu cepat dan mudah. Lebih lanjut, ciptaan dibagi ke dalam tak terhitung jumlahnya dari famili, jenis, dan spesies, dan bahkan pecahan yang lebih kecil lagi, semuanya hadir dengan jumlah yang begitu melimpah. Kendatipun demikian, kita tidak melihat apapun melainkan keteraturan, karya seni, dan ketenangan dalam ciptaan tersebut. Ini menunjukkan keberadaan sesuatu yang memiliki kekuasaan dan pengetahuan yang mutlak, yakni Allah.
6. Segala hal yang diciptakan memiliki tujuan.  Ambil contohnya pada ekologi. Segala sesuatu, bagaimanapun nampak tidak berarti, akan tetapi memiliki peran serta tujuan yang penting. Rantai penciptaan hingga pada umat manusia, yang merupakan ujungnya, sudah jelas diarahkan untuk tujuan akhir. Pohon penghasil buah memiliki tujuan untuk menghasilkan buah, dan keseluruhan hidupnya diarahkan untuk tujuan itu. Demikian pula, �pohon penciptaan� akan menghasilkan umat manusia sebagai buahnya yang paling akhir dan paling lengkap. Tidak ada satupun yang sia-sia; melainkan, segala benda, kegiatan, dan kejadian memiliki tujuan. Ini menghendaki sang bijak yang menginginkan tujuan tertentu dalam penciptaan. Karena hanya umat manusia yang dapat memahami tujuan tersebut, maka kebijaksanaan dalam penciptaan menghendaki adanya  Allah.
7. Segala makhluk hidup dan tak hidup tidak akan bisa mendapatkan kebutuhannya dengan sendirinya. Sebagai contoh, alam semesta hanya bisa bekerja dan mempertahankan eksistensinya hanya dengan adanya hukum alam semisal pertumbuhan, reproduksi, tarikan dan tolakan. Namun apa yang disebut sebagai �hukum alam� ini sama sekali tidak memiliki bentuk yang nyata, serta terlihat, berwujud kebendaan; mereka semua benda mati.  Bagaimana sesuatu yang hakikatnya adalah benda mati, yang sama sekali tidak memiliki pengetahuan dan kesadaran, akan bertanggung jawab terhadap penciptaan yang penuh keajaiban yang menginginkan kekuasaan yang mutlak serta pengetahuan yang mutlak, kebijaksanaan, kehendak, dan kecenderungan? Jadi, sesuatu yang memiliki sifat-sifat ini haruslah yang menciptakan �hukum alam� ini dan menggunakannya sebagai tirai untuk menutupi tindakannya dengan tujuan tertentu.
8. Tanaman menginginkan udara dan air, demikian pula panas dan cahaya, untuk bisa bertahan hidup. Dapatkah mereka mencukupi keinginan mereka sendiri? Kebutuhan manusia sangat tidak terbatas. Beruntungnya, segala kebutuhan kita yang paling dasar, sejak awal kita di dalam kandungan hingga kematian, disediakan oleh seseorang yang bisa menyediakannya dan memilih untuk melakukannya. Ketika kita memasuki kehidupan di dunia ini, kita mendapati bahwa segala sesuatu telah disiapkan untuk memenuhi kebutuhan indra kita dan kebutuhan intelektual dan kebutuhan spiritual kita. Ini dengan jelas menunjukkan bahwa sesuatu yang maha pemurah dan maha mengetahui memberikannya untuk segala makhluk ciptaannya dengan cara yang sangat luar biasa, dan mengakibatkan segala hal untuk saling bekerja sama untuk tujuan tersebut.
9. Segala hal dalam alam semesta, bagaimanapun jaraknya, saling membantu satu sama lain. Keadaan saling membantu satu sama lain adalah sangat menyeluruh di mana, sebagai contoh, hampir semua hal, di antaranya adalah udara dan air, api dan tanah, matahari dan langit, membantu kita dengan cara yang  biasa dan telah ditentukan sebelumnya. Sel-sel dalam tubuh kita, organ-organ, dan sistem-sistem bekerja bersama-sama untuk membuat kita tetap hidup. Tanah dan udara, air dan panas, serta bakteri saling bekerja sama untuk memberi manfaat bagi tanaman. Aktivitas tersebut, yang memperlihatkan pengetahuan dan kesadaran, yang dilakukan oleh sesuatu yang tidak hidup menunjukkan keberadaan pembuat keajaiban. Dia yang satu adalah Allah.
10. Sebelum umat manusia mulai mengotori udara, air dan tanah, segala hal di alam secara terus menerus dibersihkan dan dimurnikan. Bahkan sekarang, ia tetap mempertahankan kemurniannya yang asli di berbagai kawasan, sebagian besar karena kehidupan moderen belum mengambil tempat. Pernahkah kamu mempertanyakan kenapa alam begitu bersih? Mengapa  hutan begitu bersih, meskipun begitu banyaknya hewan yang mati di sana setiap hari? Jika semua lalat yang dilahirkan di musim panas tetap bertahan hidup,  maka permukaan bumi akan ditutupi seluruhnya dengan lapisan lalat-lalat yang mati. Tidak ada yang sia-sia di alam, karena di setiap kematian adalah awal bagi kelahiran. Sebagai contoh, jasad yang mati akan membusuk dan diserap oleh tanah. Unsur-unsur mati dan dimunculkan kembali pada tanaman; tanaman yang mati di dalam perut binatang dan manusia akan dinaikkan ke statusnya yang lebih  tinggi. Siklus kehidupan dan kematian adalah salah satu faktor yang tetap membuat alam semesta menjadi bersih dan murni, Bakteri dan serangga, angin dan hujan, lubang hitam dan oksigen dalam tubuh makhluk hidup semua mempertahankan kemurnian dari alam.  Kemurnian menunjuk pada sesuatu yang paling suci, yang sifat-sifatnya juga termasuk pada kebersihan dan kemurnian.
11. Tak terhitung jumlahnya manusia yang telah hidup sejak Adam dan Hawa diciptakan. Meskipun memiliki awal yang sama yakni pada sperma dan sel telur, dibentuk dari makanan yang sama yang dimakan oleh orang tuanya dan tersusun oleh struktur dan unsur yang sama, namun setiap orang memiliki wajah yang unik. Sains tidak bisa menjelaskan keunikan yang sangat ajaib ini. Tidak bisa dijelaskan oleh susunan kromosom atau DNA, karena perbedaan ini dimulai sejak awal pembeda-bedaan individu di dunia ini. Lagi pula, perbedaan ini tidak hanya pada wajah; semua manusia adalah unik dalam karakteristiknya, keinginannya, ambisi, dan kemampuan,  dan seterusnya. Ketika anggota dari spesies binatang adalah hampir sama semuanya dan menampakkan tidak adanya perbedaan dalam kelakuannya, namun tiap individu dari manusia seperti berada pada spesies yang berbeda yang memiliki dunia tersendiri dalam cakupan dunia umat manusia. Ini tentu saja menunjukkan dia yang satu dengan kehendak bebas yang mutlak dan pengetahuan mencakup segala hal: dialah Allah.
12. Kita membutuhkan setidaknya 15 tahun untuk mengarahkan kehidupan kita dan memahami apa yang baik dan apa yang buruk. Akan tetapi banyak binatang memiliki pengetahuan tersebut segera setelah mereka dilahirkan. Sebagai contoh, anak bebek dapat berenang segera setelah mereka menetas, dan semut mulai menggali sarangnya di tanah ketika mereka mulai meninggalkan  kepompongnya. Lebah dan laba-laba dengan cepat memahami bagaimana membuat sarang lebah dan sarang jaring laba-laba, yang merupakan karya seni yang memukau yang tidak bisa kita tiru. Siapa yang mengajarkan belut yang lahir di perairan Eropa untuk menemukan jalannya ke rumahnya di Pasifik? Bukankah perpindahan burung-burung adalah hal yang misterius? Bagaimana kamu menjelaskan fakta menakjubkan tersebut selain menganggapnya sebagai arahan atau panduan oleh Sesuatu yang mengetahui segala hal, dan telah mengatur alam semesta dan para penghuninya dengan cara di mana setiap ciptaan bisa mengarahkan kehidupannya?
13. Meskipun begitu melimpahnya kemajuan dalam ilmu pengetahuan, kita masih tetap saja tidak dapat menjelaskan kehidupan itu sendiri. Kehidupan adalah pemberian dari dia yang paling hidup, yang meniupkan roh ke dalam setiap embrio. Kita memiliki pemahaman sedikit mengenai roh dan hubungannya dengan raga, akan tetapi ketidaktahuan kita tidak berarti bahwa roh itu tidak ada. Roh dikirim di sini untuk disempurnakan dan untuk mencapai keadaan yang sesuai untuk kehidupan selanjutnya.
14. Kesadaran kita adalah pusat dari kecenderungan kita terhadap baik dan buruk. Setiap orang akan merasakan kesadaran ini kadang-kadang,  dan sebagian besar orang akan berpaling kepada Allah pada kesempatan tertentu. Bagi kita, kecenderungan ini dan keimanan kepada-Nya adalah hal yang hakiki. Bahkan jika kita secara sadar mengingkari Allah,  namun alam bawah sadar kita kadang-kadang menunjukkan keimanan kepada-Nya.  Al-Quran menyebutkan hal ini pada beberapa ayat:
Dia-lah yang membuatmu untuk berjalan di daratan dan lautan; dan ketika kamu berada di kapal, dan kapalnya berjalan akibat angin sepoi-sepoi dan mereka bergembira di dalamnya, maka akan datang pada mereka angin yang kencang, dan ombak datang pada mereka dari setiap penjuru dan mereka mengira sudah dikepung. Kemudian mereka menangis kepada Allah, membuat iman mereka murni hanya untuk-Nya, dan berkata: �jika Kamu menyelamatkan kami dari ini, kami akan sangat berterima kasih.� (Al-Quran 10:22) 
Maka [Ibrahim] menghancurkan mereka [berhala-berhala] menjadi pecahannya, seluruhnya kecuali satu yang paling besar, sehingga mereka dapat mencarinya. [Ketika mereka kembali dan melihatnya] mereka berkata: �siapa yang melakukan ini kepada tuhan kami? Sudah pasti itu orang jahat.� Mereka (yang lain) berkata, �Kami mendengar ada seorang pemuda yang mencela (berhala-berhala ini), namanya Ibrahim.� Mereka berkata, �(Kalau demikian) bawalah dia dengan diperlihatkan kepada orang banyak, agar mereka menyaksikan.� Mereka bertanya, �Apakah engkau yang melakukan (perbuatan) ini terhadap tuhan-tuhan kami, wahai Ibrahim?� Dia (Ibrahim) menjawab, �Sebenarnya (patung) besar itu yang melakukannya, maka tanyakanlah kepada mereka, jika mereka dapat berbicara.� Maka mereka kembali kepada kesadaran mereka dan berkata, �Sesungguhnya kamulah yang menzalimi (diri sendiri).� Kemudian mereka menundukkan kepala (lalu berkata), �Engkau (Ibrahim) pasti tahu bahwa (berhala-berhala) itu tidak dapat berbicara.� Dia (Ibrahim) berkata, �Mengapa kamu menyembah selain Allah, sesuatu yang tidak dapat memberi manfaat sedikit pun, dan tidak (pula) mendatangkan mudharat kepada kamu? Celakalah kamu dan apa yang kamu sembah selain Allah! Tidakkah kamu mengerti?� Mereka berkata, �Bakarlah dia dan bantulah tuhan-tuhan kamu, jika kamu benar-benar hendak berbuat.� (Al-Quran 2:58:68) 
Jadi, roh atau jiwa manusia dan kesadaran adalah dalil yang kuat bagi keberadaan Allah. 
1. Manusia memiliki kecenderungan terhadap kebaikan dan keindahan, kebajikan dan nilai-nilai moral, dan menjauhi kejahatan dan keburukan. Dengan demikian,  kecuali dikotori oleh faktor dan kondisi dari luar, kita biasanya akan mencari kebaikan yang universal dan nilai-nilai moral. Ini ternyata merupakan kebajikan dan moralitas yang sama yang diajarkan oleh semua agama langit. Sejarah sudah menyaksikan, bahwa umat manusia selalu memiliki agama tertentu untuk dipeluk. Seperti halnya tidak ada sistem lain yang bisa menggantikan agama dalam kehidupan manusia, Nabi-nabi dan orang-orang beragama adalah yang paling mempengaruhi kita dan meninggalkan jejak yang cukup membekas bagi kita. Ini merupakan bukti yang tidak terbantahkan bagi adanya Allah yang satu.
2. Kita merasakan adanya intuisi dan emosi yang merupakan pesan dari alam non materi. Di antaranya, intuisi tentang keabadian akan bangkit dalam keinginan terhadap keabadian, yang  sudah kita upayakan dengan berbagai cara. Akan tetapi, keinginan hanya bisa diwujudkan melalui keimanan dan beribadah kepada Dia yang abadi yang mengilhami kita akan hal itu. Kebahagiaan sesungguhnya dari manusia berada pada pemenuhan keinginan akan keabadian.
3.  Jika beberapa pembohong datang pada kita beberapa kali dan mengatakan hal yang sama, kita bisa saja, tanpa adanya informasi yang bisa dipercaya, percaya kepada mereka. Namun ketika puluhan ribu nabi-nabi yang tidak pernah berbohong, ratusan ribu wali-wali, dan jutaan orang-orang beriman, semuanya mengambil kejujuran sebagai pilar paling penting dari keimanan, dan semuanya setuju dengan keberadaan Allah, maka apakah masuk akal untuk menolak kesaksian mereka dan menerima berita dari beberapa orang pembohong?
4. Bukti bagi Al-Quran sebagai kitab yang diturunkan dari langit juga merupakan bukti bagi keberadaan Allah. Al-Quran mengajarkan dengan penuh penekanan dan perhatian, seperti halnya juga Al-Kitab Perjanjian Lama dan Perjanjian Baru, tentang keberadaan Allah. Tambahan lagi, puluhan ribu nabi-nabi telah dikirim untuk membimbing umat manusia kepada kebenaran. Semuanya dikenal karena kejujurannya dan sifat-sifat terpuji lainnya, dan semuanya mementingkan pada ajakan mengenai keberadaan dan keesaan Allah.
Dalil  Bagi Keesaan Allah
1. Segala sesuatu yang ada  menunjukkan Keesaan Allah. Sebagai contoh, begitu banyaknya dalil bagi keberadaan dan keesaan Allah, mari kita tinjau kehidupan: Dia menciptakan segala sesuatu dari satu hal, dan menciptakan satu hal dari banyak hal. Dia membuat tak terhitung jumlahnya sistem dan organ-organ pada tubuh binatang dari pembuahan sperma yang tersusun atas air dan cairan. Sesuatu yang mampu melakukan ini haruslah Dia yang  satu yang mutlak dan sangat berkuasa.  Sesuatu yang mengubah dengan penuh keteraturan segala unsur yang terkandung pada berbagai  jenis tumbuh-tumbuhan  atau makan binatang lainnya menjadi wujud tubuh dan anggota-anggota tubuh dari binatang itu, menggunakan bahan makanan tadi untuk menenun kulit  yang unik pada masing-masing binatang, adalah sudah pasti Dia yang paling berkuasa dan paling mengetahui.
2. Udara menggambarkan keesaan-Nya.  Sebuah pengantar yang ajaib, yang mengantarkan tak terhitung jumlahnya bunyi, suara,  gambar, dan banyak hal lainnya secara serempak, tanpa kebingungan, tanpa menghalangi yang lain. Ini menunjukkan bahwa terdapat Sesuatu, yang tanpa sekutu, yang menciptakan, dan mengendalikan, dan mengatur segala sesuatu sesuai dengan kebijaksanaan-Nya.
3. Alam semesta laksana sebuah pohon yang tumbuh dari bibit yang berisi program yang menyeluruh untuk kelangsungan hidupnya. Segala sesuatu saling terkait satu sama lain. Sebagai contoh, sebuah partikel pada pupil mata,  memiliki hubungan dan tanggung jawab terhadap mata, demikian pula dengan kepala; kekuatan untuk berkembang biak, tarikan, dan dorongan; vena dan arteri, saraf motoris dan sensoris yang mengantarkan darah dan mengoperasikan sistem di tubuh; dan dengan bagian tubuh lainnya. Ini tentu saja menunjukkan bahwa seluruh tubuh, termasuk setiap partikel,  merupakan hasil karya Dia yang abadi, dan maha kuasa, dan yang mengoperasikan segala sesuatu dengan perintah-Nya.
Sebuah molekul udara, bisa saja mengunjungi bunga dan buah-buahan apa saja, dan bekerja di dalamnya. Jika molekul pengembara ini tidak tunduk dan diatur oleh perintah dari Dia yang maha kuasa, maka molekul ini harus mengetahui seluruh sistem dan struktur dari semua tanaman dan buah-buahan, dan bagaimana mereka dibentuk, hingga ke hal yang paling detail dan paling rumit.  Jadi, molekul udara ini menunjukkan keesaan Allah seperti halnya matahari, yang berpasangan dengan cahaya, tanah, dan air. Dan seperti  yang kita tahu, ilmu pengetahuan mengatakan bahwa bahwa bahan dasar bagi segala sesuatu adalah hidrogen,  oksigen, karbon, dan nitrogen. 
Benih dari segala tanaman penghasil bunga dan buah-buahan disusun oleh hidrogen, oksigen, karbon, dan nitrogen. Mereka hanya berbeda oleh  karena adanya program yang ditanamkan padanya oleh kehendak Allah.  Jika kita menaruh  beberapa jenis bibit dalam sebuah pot bunga yang terisi tanah,  yang tentu memiliki unsur-unsur tertentu, maka tiap tanaman akan mengambil bentuk dan lekukannya  yang unik nan mengagumkan. Jika partikel-partikelnya tidak ditundukkan dan diarahkan oleh Sesuatu yang mengetahui segala hal pada tanaman baik sifat, ciri, struktur, siklus hidup, dan kondisinya;  Sesuatu yang menganugerahkan segala hal dengan apa yang cocok dan dibutuhkan olehnya; dan untuk Sebuah Kekuatan ketundukan diarahkan tanpa perlawanan, maka tidak akan timbul banyak persoalan. 
Jika dibuat lebih mudah, tanpa campur tangan Allah, maka tiap partikel dari tanah akan berisi �pabrik tersendiri� yang menentukan    segala yang akan terjadi pada tanaman tersebut. Ia juga akan membutuhkan bengkel dengan jumlah yang sama dengan jumlah tanaman penghasil bunga dan buah-buahan yang ada, sehingga di setiap bengkel itu akan menghasilkan produk-produk yang unik yang sesuai dengan keperluan tanaman tadi.   Jika tidak, maka tiap tanaman harus memiliki pengetahuan serta kekuasaan yang sangat luas sehingga dapat menciptakan dirinya sendiri. Jadi dengan  tidak adanya Allah, maka sama saja dengan mengatakan ada terdapat begitu banyak Tuhan sebanyak jumlah partikel yang ada di tanah. Yang merupakan kepercayaan yang tidak masuk akal.  
Setiap partikel berisi saksi dua yang bisa dipercaya yang merupakan syarat bagi keberadaan dan keesaan Sang Pencipta. Pertama, sang partikel dapat melakukan banyak tindakan yang cukup berarti, meskipun ia  sama sekali tidak memiliki kekuatan. Kedua, dengan bertindak  sesuai dengan keteraturan  alam semesta, maka setiap partikel akan memperlihatkan kesadaran semesta meskipun ia tidak memiliki kehidupan. Tiap partikel akan diuji melalui ketidakmampuannya sehingga membutuhkan adanya Suatu Kekuatan Yang Maha Kuasa, dan dengan bertindak sesuai dengan keteraturan semesta terhadap keesaan-Nya. 
1. Setiap orang adalah bentuk mini dari alam semesta, buah dari pohon penciptaan dari alam semesta; dan benih dari dunia ini, karena setiap dari kita akan terdiri oleh sampel dari semua makhluk hidup lainnya. Itu seperti halnya jika setiap orang adalah tetesan yang disuling dari alam semesta, yang memiliki keseimbangan yang paling halus dan paling sensitif. Untuk menghasilkan makhluk hidup yang seperti itu dan untuk bisa menjadi Tuhannya menghendaki adanya kendali terhadap seluruh alam semesta.
2. Dengan hal ini, kita dapat mengerti bahwa hal-hal yang berikut mewakili stempel yang unik dari Pencipta segala hal, Penguasa yang maha agung dari alam semesta: membuat seekor lebah sebagai petunjuk bagi banyak hal; menuliskan banyak sifat-sifat alam semesta pada diri manusia;  termasuk program bagi siklus kehidupan pohon beringin dalam bibit kecil dari pohon beringin; memperlihatkan kerja dari nama-nama-Nya yang terwujudkan di seluruh alam semesta pada hati manusia; dan terekam dalam ingatan kita, ditempatkan pada tempat yang sanga kecil, namun memiliki informasi yang cukup untuk memenuhi seluruh perpustakaan, serta segala daftar kejadian di seluruh alam semesta.
3. Segala kehidupan adalah sebuah simfoni dari saling tolong menolong satu sama lain. Seperti juga organ dan anggota tubuh yang hidup, sistem-sistem dan sel-sel, maka segala bagian dari alam semesta saling menopang satu sama lain. Sebagai contoh, udara dan air, tanah dan matahari, bekerja sama sehingga memungkinkan bagi sebuah apel bisa hadir menjadi ada. Seperti halnya bagian-bagian dari sebuah pabrik atau batu penyusun sebuah tempat, segala ciptaan saling mendukung dan menolong satu sama lain, dan bekerja sama untuk kebutuhan satu sama lain dalam keteraturan yang begitu sempurna. Dengan usaha bersama, mereka melayani makhluk hidup. Unsur-unsur pada tanah membantu tanaman untuk membuatnya bisa hadir dan bertahan hidup. Sebagian besar binatang hidup bergantung pada tanaman, dan manusia hidup bergantung pada tanaman dan binatang. Jadi, unsur-unsur akan membentuk fondasi dasar bagi penyusun dari bentuk fisik makhluk hidup.
Dengan tunduk pada tindakan dari aturan ini yang saling menopang, yang diterapkan di seluruh alam semesta dari matahari dan bulan, malam dan siang, musim panas dan musim dingin, bagi tanaman yang menopang binatang yang kelaparan dan membutuhkannya, binatang menopang manusia, nutrisi menopang kebutuhan bayi, kemudian partikel-partikel  pada buah-buahan dan tanaman menopang sel-sel di tubuh, itu  menunjukkan bahwa mereka bertindak berdasarkan kekuasaan sesuatu yang tunggal, Pengasuh yang maha pemurah, dan dalam perintah dari Dia yang satu, Pengatur yang paling bijaksana. 
1. Pemeliharaan alam semesta dan kebaikan dari kebijaksanaan dari alam semesta nampak jelas dalam setiap ciptaan yang bermanfaat.  Ini, bersamaan dengan anugerah yang penuh dengan kemurahan serta kelangsungan alam semesta yang dibutuhkan oleh anugerah tersebut memberikan segala makhluk hidup makanan,  membentuk stempel bagi keesaan Allah dengan begitu menakjubkan di mana setiap orang bisa melihat dan memahaminya.
Segala makhluk, khususnya yang sudah hidup, harus memenuhi keinginan dan kebutuhannya agar tidak bisa hidup. Ini berlaku entah makhluk yang dimaksud adalah seluruhnya atau hanya sebagian, sebuah individu atau spesies. Akan tetapi mereka tidak dapat memenuhi bahkan kebutuhannya yang paling kecil. Malahan, semua kebutuhan mereka didapatkan dengan cara yang tidak terduga dan dari tempat yang tidak terduga juga, dengan pemilihan waktu dan instruksi yang tepat, dengan cara yang sesuai dan dengan kebijaksanaan yang sempurna. Semua ini menunjukkan keberadaan dari Sesuatu yang maha bijaksana dan maha mulia, Pemberi anugerah yang meliputi segala hal. 
1. Mari kita tinjau matahari. Mulai dari planet hingga tetes-tetes air, pecahan gelas, dan butiran salju yang mengkilau, pancaran cahaya dari matahari nampak pada mereka. Jika kamu tidak menyetujui bahwa matahari kecil yang  terlihat pada segala hal ini  hanyalah pantulannya, maka kamu harus menyetujui keberadaan matahari pada tiap tetes air, pecahan gelas dan setiap benda transparan yang menghadap ke arah matahari. Apakah ini masuk akal?
Jika gambaran pantulan tersebut tidak katakan berasal dari matahari, maka kamu harus menerima adanya matahari dalam jumlah banyak sebagai pengganti matahari yang satu. Apakah  ini logis? Demikian pula,  jika segala sesuatu tidak dikaitkan dengan keberadaan Allah yang satu, yang memiliki kekuasaan meliputi segala hal, maka kamu harus menerima bahwa terdapat  bayak Tuhan di sana sebanyak partikel di alam semesta. Bagaimana kamu bisa menerima hal tersebut? 
1. Sepanjang musim panas dan musim dingin, Allah menghidupkan tidak terhitung jumlahnya tanaman dan spesies binatang, tiap anggotanya adalah unik. Prosesnya begitu teratur di mana tidak ada satupun kebingungan meskipun begitu banyaknya percampuran di sana. Dia �menuliskan� di wajah bumi individu-individu dari spesies yang tak terhitung jumlahnya tanpa kesalahan dan tanpa kelupaan, kekeliruan atau ketidakcukupan. Semuanya dilakukan dengan cara paling seimbang, paling proporsional,  paling teratur, dan dengan sangat sesuai. Ini menunjuk pada Satu hal yang paling kuasa dari yang paling sempurna, paling bijak dan paling pemurah dari yang paling indah, sesuatu yang memiliki kekuasaan yang tidak terbatas, pengetahuan yang begitu luas, dan mampu mengatur seluruh alam semesta.
Coba kita lihat apa yang terjadi pada saat musim dingin dan musim panas. Banyaknya campur tangan Allah pada kedua musim tersebut sangat ajaib dalam kaitannya dengan jangkauannya, kecepatannya, dan kebebasannya, demikian halnya dalam hal jumlah dan urutannya, keindahan dan penciptaannya. Hanya Dia yang satu dengan pengetahuan tidak terbatas  serta kekuasaan yang sangat luas yang bisa memiliki �stempel�  semacam itu. Stempel semacam hanya akan bisa diberikan kepada Dia yang satu yang berada di mana saja meskipun dia tidak berdiam di manapun, maha hadir dan maha melihat, tidak ada satupun yang bisa bersembunyi dari-Nya juga tidak satupun yang sulit bagi-Nya, dan semua partikel-partikel dan bintang-bintang adalah setara dengan kekuasaan-Nya. 
1.  Benih yang ditaburkan di pekarangan akan menunjukkan bahwa pekarangan dan benih tersebut berada pada pemiliknya. Demikian pula, unsur-unsur fundamental dari kehidupan (misalnya udara, air, dan tanah) adalah hadir di mana-mana meskipun kesederhanaannya dan sifat-sifatnya yang sama. Tanaman dan binatang ditemukan di mana-mana, kendatipun hakikatnya mereka memiliki sifat-sifat yang sama yang berlawanan dengan beragamanya kondisi dari kehidupan.
Semua ini dikendalikan oleh Sang Pembuat yang penuh keajaiban. Setiap tanaman, buah-buahan, dan binatang adalah stempel, atau sebuah bendera, atau tanda tangan dari Sang Pembuat tersebut. Di manapun mereka ditemukan, mereka akan mengatakan dengan lidahnya: �Sesuatu yang menjadikanku stempel Dia lah yang menciptakan tempat ini. Sesuatu yang menjadikanku bendera Dia lah yang memiliki tempat ini. Sesuatu yang menjadikanku tanda tangan Dia lah yang menyulam tanah ini.� Dengan kata lain, hanya Sesuatu yang menggenggam seluruh unsur-unsur dengan Kekuasaan-Nya lah yang bisa memiliki dan mempertahankan kehidupan paling kecil sekalipun. Siapa saja bisa melihat bahwa hanya Dia yang memiliki kekuasaan terhadap semua jenis tanaman dan binatang dapat memiliki, mempertahankan, dan mengatur bahkan yang paling sederhana dari mereka. 
Sungguh, dengan lidah yang sama dengan lidah yang lainnya, setiap individu akan berkata: �hanya Sesuatu yang memiliki spesies ku lah yang bisa memilikiku.� Pada lidah yang terikat dengan matahari, dan hubungan yang saling membutuhkan dengan langit, bumi, dan planet-planet lainnya akan berkata: �hanya Sesuatu yang bisa memiliki semua ini lah yang bisa memilikiku.� Jika sebuah apel bisa memiliki kesadaran dan seseorang berkata ke salah satu dari mereka: �Kamu adalah hasil karya seniku,� sang apel akan membalas: �Diamlah! Jika kamu bisa menciptakan seluruh apel, atau setidaknya jika kamu bisa membuang dengan mudah segala pohon penghasil buah di planet ini dan semua pemberian dari Dia yang maha pemurah yang mendatangkannya dari anugerahnya yang mulia, dalam satu muatan kapal, maka hanya dengan itu kamu bisa mengatakan bahwa kamu pemilikku.� 
Karena setiap buah-buahan bergantung pada satu hukum pertumbuhan dari Satu Pusat, maka mudah pula untuk menghasilkan satu atau banyak  buah lainnya. Dengan kata lain,  agar beberapa pusat bisa menghasilkan bisa menghasilkan satu biji buah itu sama mahal dan sulitnya seperti mempersiapkan sebuah pohon, dan untuk menghasilkan peralatan yang dibutuhkan oleh seorang prajurit akan membutuhkan seluruh pabrik yang digunakan untuk memperlengkapi seluruh  pasukan.  Intinya adalah: ketika sebuah hasil terhubung dengan banyak individu yang bergantung pada beberapa pusat, maka akan ada banyaknya kesusahan sebanyak individu yang dilibatkan. Jadi, kemudahan yang mengagumkan yang terlihat pada banyak spesies muncul dari adanya kesatuan. 
Hubungan dan kemiripan dalam beberapa sifat-sifat dasar dan bentuk-bentuk dasar yang terlihat pada semua anggota dari spesies, dan dalam pembagian genus, merupakan bukti bahwa mereka adalah hasil karya Sang Pencipta yang satu, karena mereka �diukir� dengan pena yang sama dan menggunakan stempel yang sama.  Kemudahan yang teramati dalam kemunculan mereka menjadi ada menghendaki bahwa mereka adalah karya Sang  Pencipta yang satu. Jika tidak, maka itu adalah hal yang sulit untuk menjadikan mereka ada sehingga genus dan spesies yang dimaksud tidak akan pernah hadir. 
 Kesimpulannya: Ketika dihubungkan dengan Allah yang maha suci, semua hal menjadi mudah layaknya satu kesatuan; ketika dihubungkan dengan sebabnya, satu hal saja bisa menjadi begitu sulit layaknya segala hal. Sebagai akibatnya, kemurahan dan kemudahan yang dijumpai di alam, juga jumlahnya yang melimpah, menunjukkan adanya stempel keesaan. Jika kelimpahan dan kemurahan dari buah-buahan tidak dimiliki oleh Satu Yang Tunggal, maka kita tidak akan dapat membelinya bahkan jika memberikan seluruh dunia. Bagaimana kita bisa membayar hubungan yang memiliki kesadaran dan penuh tujuan antara tanah dan udara, air dan matahari,  panas matahari dan bibit, dan banyak hal lainnya yang membuat kehadiran dari sebuah delima  menjadi mungkin? Semua faktor ini adalah sadar sifatnya dan dikendalikan oleh Sang Pencipta Yang Satu, yakni Allah Yang Maha Suci. Ongkos bagi sebuah delima atau buah-buahan lainnya adalah seluruh alam semesta. 
1. Kehidupan, yang merupakan perwujudan kemurahan Allah, adalah dalil dan bukti bagi keesaan Allah, juga merupakan perwujudan dari-Nya.  Kematian, yang merupakan perwujudan Keagungan Allah, adalah dalil dan bukti bagi Keesaan-Nya.
Sebagai contoh, gelembung pada permukaan sungai akan memperlihatkan gambar matahari, cahaya, dan pantulannya, seperti halnya semua benda-benda yang transparan. Fakta-fakta ini merupakan saksi bagi keberadaan matahari. Kendatipun gelembung-gelembung kadang-kadang menghilang (misalnya ketika mereka lewat di bawah jembatan), kelanjutan dari pantulan matahari yang begitu indah serta cahayanya yang tampak pada gelembung-gelembung berikutnya membuktikan bahwa gambar matahari (yang nampak, menghilang, dan kemudian muncul lagi) datang dari keabadian pada matahari yang muncul dari ketinggian. Dengan demikian, penampakan dari gelembung-gelembung yang berkilauan ini menunjukkan adanya keberadaan matahari, dan lenyapnya menunjukkan kesatuan dan kesinambungan. 
Dengan cara yang sama, berada dalam aliran yang terus-menerus yang disaksikan melalui keberadaan dan kehidupannya bagi keharusan akan keberadaan dan keesaan oleh suatu zat yang mesti ada. Mereka menyaksikan keesaan-Nya, keabadian-Nya, dan kekekalan-Nya melalui kematian makhluk-makhluk-Nya. Keindahan dari makhluk-Nya yang begitu halus yang diperbaharui dan direkrut, beserta dengan pergantian siang dan malam, juga pergantian musim, dan perjalanan waktu menunjukkan keberadaan,  keesaan, 


Dalil  Yang Salah Tentang Asal Mula dari Keberadaan (Eksistensi)
Pemahaman orang-orang abad pertengahan mengenai keberadaan dan sifat-sifat alam semesta telah didukung oleh otoritas Gereja Katolik. Gereja, yang bersandar pada Wahyu yang diturunkan (Alkitab) yang sudah mengalami perubahan seiring berjalannya waktu, memandang sains moderen sebagai ancaman terhadap kewenangannya, dan memandangnya dengan permusuhan. Keretakan antara sains dan agama makin mendalam hingga keduanya menjadi tidak bisa dipertemukan. Pada akhirnya, agama berpindah ke dalam domain keimanan yang buta dan ritual yang menghibur dianggap asing bagi ilmu pengetahuan. Jadi, sains tidak lagi bisa tunduk pada otoritas keagamaan. Penjelasan Darwin mengenai evolusi menyegel dan mempopulerkan ide bahwa keberadaan kita adalah bermula dengan sendirinya dan berlangsung dengan sendirinya, sebuah proses yang dibuka dengan sendirinya berdasarkan hukum yang suatu saat nanti akan bisa dipahami sepenuhnya (dan kemudian hingga derajat tertentu dapat dimanipulasi) oleh umat manusia. 
Tidak semua ilmuwan yang menyetujui bahwa sebab dari alam atau apa yang disebut sebagai hukum alam bisa menjelaskan semua fenomena. Sebelum membahas masalah ini, kita harus menekankan bahwa seluruh nabi, tidak peduli waktu dan tempatnya, setuju tentang bagaimana keberadaan bermula dan bagaimana itu terus berlangsung dan dipertahankan, dan segala hal lainnya yang berhubungan dengan kehidupan dan keberadaan (eksistensi). Ketika sejumlah besar ilmuwan setuju dengan para Nabi, beberapa ilmuwan dan filsuf yang lebih memilih naturalisme dan materialisme memiliki perbedaan yang sangat besar dalam penjelasannya. Beberapa menghubungkan kreatifitas dan keabadian, demikian halnya kehidupan dan kesadaran, semata-mata dengan materi. Yang lainnya menggunakan dalil bahwa alam sifatnya abadi dan hadir dengan sendirinya dan segala sesuatu bisa dijelaskan dengan sebab alami dan dengan menggunakan hukum alam. Sementara ada pula, karena tidak bisa menjelaskan asal mula kehidupan, jatuh pada gagasan semacam kebetulan dan keharusan. 
Poin-poin berikut akan menunjukkan betapa kemustahilan untuk menjelaskan eksistensi tanpa menyetujui keberadaan dan keesaan Allah. 
Alam, Hukum Alam dan Sebab 
1. Hukum alam sifatnya hanya pada nama saja (nominal) namun tidak memiliki bentuk yang aktual. Mereka sifatnya adalah proposisi yang ditawarkan sebagai penjelasan dari peristiwa-peristiwa atau fenomena-fenomena tertentu,  dan menyebut akan adanya gaya imajiner yang disimpulkan dari pergerakan atau hubungan antara tiap kejadian atau fenomena. Hukum gravitasi, perkembangbiakan dan pertumbuhan dari makhluk hidup, tarikan dan tolakan magnetis, serta lainnya bukanlah suatu entitas yang dapat dipastikan melalui indra kita atau dengan menggunakan perkakas sains.  Sebagai contoh, bagaimanapun benarnya hukum gravitasi, dapatkah kita menghakimi bahwa alam semesta (sesuatu di mana hukum ii bekerja) bisa hadir karena adanya hukum tersebut? Adalah beralasan untuk menganggap bahwa akibat dari  keberadaan dari segala hal, bahkan kecerdasan dan kesadaran dari makhluk hidup, adalah proposisi-proposisi?
2.  Hukum alam dan sebab-sebab di alam disimpulkan dari pergerakan atau hubungan antara kejadian atau fenomena yang teramati di alam semesta. Dengan demikian, karena mereka bergantung pada faktor eksternal, mereka sama sekali tidak bisa dikatakan mandiri juga tidak bisa dikatakan hadir dengan sendirinya.
3.  Keberadaan alam semesta, demikian pula peristiwa dan fenomena sifatnya tidak pasti (kontingen). Jadi tidak ada satupun padanya yang harus ada,  karena sama saja kemungkinannya  untuk ada atau tidak ada. Terdapat hampir tidak terhingga jumlahnya sel-sel pada embrio yang bisa dikunjungi oleh partikel makanan. Sesuatu yang keberadaannya tidak pasti ada atau tidaknya tidak bisa bersifat abadi, karena seseorang bisa saja lebih memilih keberadaannya ketimbang ketidakberadaannya atau sekedar kemungkinannya untuk ada.
4. Sebagaimana hal lainnya yang tidak pasti adanya di mana harus berada di ruang dan waktu tertentu, mereka semua harus memiliki awal. Sesuatu yang memiliki awal sudah pasti akan memiliki akhir, jadi dengan demikian tidak bisa bersifat abadi.
5. Sebab di alam saling membutuhkan satu sama lain untuk bisa memberikan hasil. Sebagai contoh, apel membutuhkan  bunga apel untuk bisa hadir, bunga membutuhkan dahan, dahan membutuhkan pohon, dan seterusnya, seperti halnya bibit membutuhkan tanah, udara, dan kelembapan, untuk bisa bertunas dan tumbuh. Setiap sebab juga bertindak sebagai hasil, kecuali jika kita menerima ada banyak Tuhan di sana sebanyak jumlah sebab, maka kita harus melihat pada suatu sebab yang tunggal di luar dari rantai sebab dan hasilnya (sebab dan akibat).
6. Agar sebuah hasil bisa hadir, ada tak terhingga jumlahnya sebab yang harus bekerjasama dalam suatu cara yang terkoordinasi dan bisa dipercaya di mana mereka bisa disebut sebagai �hukum alam.� Coba tinjau ini: agar supaya bisa hadir, maka sebuah apel membutuhkan saling kerjasama antara udara dan tanah, matahari dan air, serta kemiringan 23 derajat dari sumbu  bumi, dan aturan yang begitu kompleks mengenai pembuahan dan pertumbuhan bagi bibit dan tanaman. Dapatkah sebab-sebab serta hukum-hukum yang sifatnya buta, dan tuli, bodoh dan tidak sadar datang bersama-sama dengan sukarela untuk menciptakan makhluk hidup? Apakah kamu benar-benar berpikir bahwa mereka bisa membentuk diri manusia, di mana semuanya adalah hidup dan memiliki kesadaran, cerdas dan bertanggung jawab, dan dapat menjawab pertanyaan tentang niat dan tindakannya?
7. Sebuah bibit kecil berisi pohon yang besar. Manusia, sebuah ciptaan yang teramat rumit, tumbuh dari sebuah sel telur yang dibuahi oleh sperma yang ukurannya mikroskopis. Adakah hubungan yang tepat atau perbandingan yang bisa diterima antara sebab dan akibat di sini? Dapatkah sesuatu yang sangat lemah dan sederhana, bodoh dan tidak hidup akan memberikan hasil yang sangat kuat dan kompleks, cerdas dan memiliki pengaruh besar bagi kehidupan?
8. Semua fenomena dan proses di alam memiliki kebalikannya: misalnya utara dan selatan, positif dan negatif, panas dan dingin,  keindahan dan keburukan, siang dan malam, tarikan dan dorongan, membeku dan mencair, menguap dan mengembun, dan seterusnya.  Sesuatu yang memiliki lawan atau kebalikan, yang menginginkannya lawannya agar bisa ada,  tidak mungkin lah bertindak sebagai sang pencipta atau dia yang paling awal.
9. Meskipun segala segala sebab yang diperlukan bagi adanya hasil telah dihadirkan, namun hasil tersebut tidak selalu muncul seperti yang diharapkan. Sebaliknya, sesuatu  bisa saja terjadi dan hadir keberadaannya tanpa adanya sebab yang bisa kita kenal dan kita pahami. Juga, penyebab yang sama tidak selalu menimbulkan hasil yang sama. Inilah sebabnya beberapa ilmuwan menolak kausalitas sebagai cara menjelaskan berbagai hal dan peristiwa.
10. Di antara berbagai sebab, manusia adalah yang paling mampu dan paling menonjol, karena kita dibedakan oleh adanya kecerdasan,  kesadaran, tekad, dan banyak kemampuan lain serta indra luar dan dalam serta perasaan. Namun kita masih saja sangat lemah dan tidak berdaya di mana bahkan sebuah mikroba dapat menimbulkan sakit yang teramat sangat bagi kita. Jika bahkan kita tidak memiliki bagian dari kehadiran kita, dan bahkan tidak memiliki kendali terhadap kerja dalam tubuh kita, bagaimana sebab lainnya bisa menimbulkan suatu kreativitas?
11. Kaum materialis menggunakan hubungan dari peristiwa sebagai hubungan sebab dan akibat. Jika dua peristiwa  hadir secara bersamaan, mereka akan membayangkan bahwa satu akan menyebabkan lainnya. Untuk mengingkari adanya Sang Pencipta mereka akan mengemukakan pendapat semisal: �air akan menyebabkan tanaman untuk tumbuh.� Mereka tidak pernah bertanya bagaimana air bisa mengetahui apa yang akan dikerjakan, dan bagaimana mengerjakannya, dan bagaimana ini bisa menghasilkan itu, dan apa kualitas yang dimilikinya sehingga mengakibatkan tanaman menjadi tumbuh?
Apakah air memiliki pengetahuan dan kekuatan untuk menumbuhkan tanaman? Apakah ia mengetahui hukum atau sifat-sifat dari pembentukan tanaman? Jika menghubungkan pertumbuhan tanaman dengan hukum yang ada di alam, apakah mereka bisa mengetahui bagaimana membentuk tanaman tersebut? Terdapat pengetahuan tertentu, keinginan, dan kekuatan  yang diperlukan untuk membuat bahkan hal yang paling kecil. Dengan demikian, bukankah Sesuatu yang memiliki pengetahuan yang maha luas, dan memiliki keinginan dan kekuasaan yang mutlak, diperlukan untuk menciptakan alam semesta yang ajaib, menakjubkan serta sangat rumit, di mana kita hanya memiliki pengetahuan yang sangat sedikit terhadapnya. 
Tinjau sekuntum bunga. Dari mana keindahannya berasal, siapa yang merancang hubungan antara ia dengan indra penciuman kita, serta penglihatan dan kemampuan untuk memaknai? Dapatkah sebuah bibit, tanah atau cahaya matahari, yang mana semuanya tidak memiliki kesadaran, kekuasaan, atau keinginan bahkan hanya untuk membuat sekuntum bunga, apalagi untuk membuatnya terlihat indah? Dapatkah kita, beserta planet ini sebagai sesuatu yang sadar, bisa membuat sekuntum bunga? Sekuntum bunga hanya bisa hadir jika seluruh alam semesta sudah hadir lebih dulu. Untuk menghasilkan sekuntum bunga, dengan demikian, seseorang harus bisa menciptakan seluruh alam semesta. Dengan kata lain,  penciptanya harus memiliki kekuasaan yang mutlak, serta pengetahuan, dan kehendak. Semua ini hanya bisa dikaitkan dengan Allah semata. 
Materi Dan Peluang
Dalil kita terhadap hukum dan sebab di alam yang sifatnya hadir  dengan sendiri, tercukupi dengan sendirinya, dan bahkan pada tahap tertentu abadi, bisa dibenarkan bagi pandangan yang menganggap bahwa kreativitas berasal dari peluang dan materi. 
Entah itu didefinisikan berdasarkan prinsip-prinsip fisika klasik atau fisika moderen, materi tentu saja bisa berubah dan rentan terhadap pengaruh dari luar. Jadi itu tidak mungkin bersifat abadi atau muncul dengan sendirinya.  Juga, karena materi buta dan tuli, tidak bernyawa dan bodoh, tidak memiliki kekuatan dan tidak memiliki kesadaran, bagaimana mungkin itu bisa menjadi asal bagi pengetahuan, kekuatan, dan kesadaran? Sesuatu tidak bisa menerapkan ke yang lain apa yang tidak dimilikinya sendiri. 
Ada begitu melimpah bukti tentang susunan, organisasi, dan harmoni yang memiliki tujuan di alam yang sama sekali tidak  masuk akal untuk dianggap sebagai kemungkinan atau kebetulan sebagai penyebabnya. Sebagai contoh, tubuh manusia tersusun dari satu juta protein. Kemungkinan bagi sebuah protein untuk terjadi secara kebetulan adalah sangat kecil. Tanpa kehadiran sesuatu yang lebih cenderung akan keberadaannya dan kemudian menciptakannya; siapa yang memiliki pengetahuan yang mutlak dan menyeluruh untuk bisa mengatur hubungannya dengan protein-protein lain, sel-sel, dan seluruh bagian dari tubuh; dan menempatkannya pada tempat yang seharusnya,  sebuah protein bisa saja tidak eksis. Sains hanya bisa menempuh jalurnya yang benar hanya ketika para pelakunya mengakui bahwa Allah yang satu sebagai pencipta segala sesuatu. 
Eksperimen sederhana berikut bisa membantu kita untuk memahami dalil yang penting ini: 
Overbeck dan rekan kerjanya di Baylor College of Medicine di Houston sedang berusaha untuk mempraktikkan terapi genetik dengan melihat apakah mereka bisa mengubah tikus albino menjadi tikus berwarna. Para peneliti berusaha menyuntikkan gen yang sangat penting bagi produksi melanin ke dalam sebuah sel dari embrio tikus albino. Mereka kemudian memberi roti pada anak-anak tikus yang dihasilkan, di mana setengahnya membawa gen pada satu bagian kromosom dari tiap pasang kromosom.  Dengan menerapkan teori Mendel tentang reproduksi akan diperoleh bahwa sekitar seperempat dari cucunya akan membawa gen pada kedua kromosom atau bersifat homozigot, atau dalam bahasa genetik haruslah berwarna. 
Namun sang tikus tidak pernah memiliki kesempatan untuk memperoleh warna.  Hal yang pertama teramati adalah, kata Overbeck, terjadi kematian pada cucu-cucu dari tikus hingga 25 persen seminggu setelah mereka dilahirkan. Penjelasannya adalah: 
Gen-gen melanin yang disuntikkan ke dalam embrio tikus albino ternyata memasukkan dirinya pada gen-gen yang sama sekali tidak berhubungan. Rantai DNA yang tidak biasa yang muncul di tengah-tengah dari gen merusakkan kemampuan dari gen tersebut untuk membuatnya bisa dibaca. Terkadang, ginjal dan pankreas juga rusak, dan sepertinya kerusakannya lah yang menjadi sebab bagi terbunuhnya tikus-tikus. 
Overbeck dan rekannya sudah berhasil menemukan lokasi gen pada kromosom tikus dan kemudian berusaha menjabarkan strukturnya. Itu akan memberikan mereka informasi mengenai struktur dari protein yang di simpan dalam gen,  bagaimana protein itu bekerja, dan di mana itu dihasilkan saat gen kemudian muncul, apakah gen muncul di seluruh bagian  tubuh, atau hanya di bagian kiri saja atau di bagian kanan saja dari embrio yang bersangkutan. Overbeck heran, dan kapan gen itu dimunculkan? 
Pertanyaan ini kemudian membawa Overbeck kepada sebuah eksperimen mengenai transfer gen. Kami pikir terdapat setidaknya 100 ribu gen, katanya, jadi kemungkinan ini terjadi berada dalam satu di antara 100 ribu. 
Harus membutuhkan ribuan uji coba, dengan demikian ribuan pula tikus, agar eksperimen tersebut bisa berhasil. Akan tetapi tidak ada uji coba seperti itu di alam. Setiap pohon yang ditempatkan di tanah kemudian bertunas dan menghasilkan pohon, kecuali ada yang mencegah itu terjadi. Demikian pula, embrio dalam kandungan tumbuh menjadi makhluk hidup yang sadar yang dilengkapi dengan kemampuan spiritual dan kemampuan intelektual. 
Tubuh manusia merupakan sebuah keajaiban bagi kesimetrisan dan ketidaksimetrisan. Ilmuwan mengetahui bagaimana itu berkembang di kandungan. Apa yang tidak mereka pahami adalah bagaimana partikel-partikel penyusunnya bisa mencapai embrio dan membedakan mana kiri dan mana kanan, bisa menentukan tempat tertentu pada organ, memasukkan dirinya di tempat yang sesuai, dan memahami hubungan yang begitu kompleks serta kebutuhan dari setiap sel dan organ. Proses ini begitu kompleksnya sehingga jika sebuah partikel yang dibutuhkan oleh pupil mata bagian kanan tiba-tiba sampai di telinga, maka embrio akan rusak dan kemudian mati. 
Sebagai tambahan, segala makhluk hidup terbuat dari unsur-unsur yang sama pada tanah, udara, dan air. Mereka sangat mirip satu sama lain dalam hubungannya dengan organ-organ dan anggota-anggota tubuhnya. Dan mereka sangat unik seluruhnya dalam kaitannya dengan kelengkapan tubuh, karakteristik, kehendak, dan ambisi. Kekhasan ini sangat bisa diandalkan sehingga kamu bisa langsung diidentifikasi secara pasti hanya dengan sidik jarimu. 
Bagaimana kita bisa menjelaskan ini? Ada dua alternatif: bisa jadi tiap partikel membawa pengetahuan, kehendak, dan kekuasaan yang tidak terbatas; atau Sesuatu yang memiliki pengetahuan, kehendak, dan kekuasaan tersebut yang menciptakan dan mengatur setiap partikel.  Akan tetapi jauh ke belakang kita pergi dalam usaha  mengubungkannya dengan sebab dan akibat dan keturunan, maka kedua penjelasan ini masih valid. 
Bahkan jika keberadaan alam semesta dihubungkan dengan sesuatu yang bukan Allah (misalnya evolusi, sebab akibat, alami, materi, atau peluang dan keharusan), kita tidak bisa mengingkari satu fakta: segala sesuatu menunjukkan, meskipun itu hadir dan bertahan dan mati, pengetahuan yang mencakup segala hal serta kekuasaan dan kehendak yang mutlak. Seperti yang kita lihat pada eksperimen Overbeck, satu saja gen yang salah tempat atau salah arah akan mengakibatkan kekacauan atau bahkan bisa mengakhiri kehidupan. Saling terhubungnya segala hal, mulai dari galaksi sampai atom, adalah kenyataan di mana setiap entitas masuk ke dalamnya dan mengetahui fungsi dan tempatnya yang unik. 
Apakah ada pertunjukan yang lebih bagus bagi keberadaan dan pelaksanaan secara bebas dari pengetahuan yang mencakup segala hal, serta kehendak dan keinginan yang mutlak, di mana partikel-partikel yang memiliki susunan biokimia yang sama akan menghasilkan -- melalui pengaturan yang begitu halus dalam hubungannya satu sama lain --  entitas yang unik serta organisme? Apakah cukup memuaskan untuk menjelaskan ini pada keturunan dan kebetulan, mengingat bahwa penjelasan seperti itu akan bersandar pada Sesuatu yang memiliki pengetahuan yang luas,  memiliki kekuasaan serta kehendak yang mutlak? 
Kita tidak boleh disesatkan oleh fakta yang nampak bahwa segala sesuatu terjadi berdasarkan suatu program, rencana atau proses tertentu. Hal semacam itu adalah tirai yang dihamparkan sepanjang alam semesta sebuah rangkaian peristiwa yang terus bergerak. Hukum alam bisa saja disimpulkan dari proses -proses ini, namun mereka sama sekali bentuk yang nyata. Kecuali jika kita menghubungkannya dengan alam (atau pada materi atau kebetulan dan kemestian) apa yang seharusnya kita hubungkan dengan Sang penciptanya, maka kita harus menerima bahwa itu, pada hakikatnya dan kenyataannya, mekanisme pencetakan dan bukan mesin cetaknya, sebuah rancangan dan bukan sang perancang, penerima yang pasif dan bukan agen, aturan dan bukan sang pengatur, kumpulan hukum- hukum yang hanya namanya saja dan bukan sebuah kekuasaan.
Untuk memahami secara lebih baik mengapa ini tidak memiliki bagian pada keberadaan, mari kita telaah tujuan, harmoni, dan saling ketergantungan pada ciptaan dengan mengamati fakta yang sederhana. Sekali lagi, Marrison menarik perhatian kita kepada beberapa hal berikut: 
Bola bumi sekarang sudah bisa dijelaskan dengan dimensinya yang permanen dan masanya sudah bisa ditentukan.  Kecepatannya dalam mengorbit mengelilingi matahari bisa dikatakan tidak berubah. Rotasi pada sumbunya sudah ditentukan secara akurat di mana adanya variasi dalam detik dalam satu abad akan mengacaukan perhitungan astronomi. Andaikan bola bumi lebih besar atau lebih kecil, atau andaikan kecepatannya sedikit berbeda, maka itu akan lebih jauh atau lebih dekat dengan matahari, dan kondisi ini akan berpengaruh terhadap kehidupan segala jenis makhluk, termasuk manusia. 
Bumi berputar pada porosnya dalam dua puluh empat jam atau dalam laju seribu juta mil per jam. Anggap itu berputar dengan kecepatan ratusan mil per jam. Mengapa tidak? Maka siang dan malam kita akan sepuluh kali lebih lama dari sekarang. Maka panasnya musim panas akan membakar tumbuh-tumbuhan di setiap hari yang panas dan setiap tunas akan membeku di malam hari. Matahari, yang menjadi sumber kehidupan, memiliki suhu permukaan sekitar 12 ribu derajat Fahrenheit, dan bumi kita berada pada posisi cukup jauh sehingga api abadi ini bisa cukup menghangatkan namun tidak terlalu panas. Jika suhu rata-rata di bumi berubah cukup drastis hingga lima puluh derajat  secara rata-rata tiap tahun, maka semua tumbuh-tumbuhan akan mati, dan manusia akan terpanggang dan membeku bersamanya. Bumi mengorbit matahari dengan laju delapan belas mil per detik. Jika laju revolusi katakanlah enam atau  empat puluh mil per detik, maka kita akan terlalu jauh atau terlalu dekat dengan matahari sehingga tidak mungkin bentuk kehidupan kita bisa hadir.
Bumi miring pada sudut 23 derajat. Ini memberikan kita musim. Jika itu tidak dimiringkan, maka kutub-kutub akan berada dalam mengalami senja yang abadi. Air yang menguap di samudra akan akan berpindah ke utara dan selatan, akan menghadirkan benua-benua yang penuh dengan es dan menyisakan gurun antara khatulistiwa dan es. 
Bulan berada pada jarak  240 ribu mil, maka pasang surut terjadi dua kali sehari yang menjadi penanda kehadirannya. Pasang bisa terjadi sampai ketinggian 50 kaki pada beberapa tempat, dan bahkan kerak bumi mengalami kelengkungan dua kali sehari akibat tarikan dari bulan. Jika bulan kita katakanlah 50 ribu mil jauhnya alih-alih berada pada jaraknya yang sekarang, maka pasang yang terjadi akan begitu dahsyatnya sehingga dalam dua kali sehari maka daerah-daerah yang rendah pada setiap benua akan berada di bawah air yang  menghempas dengan kencang bahkan gunung-gunung akan mengalami pengikisan,  dan mungkin saja tidak ada benua yang akan muncul dari kedalaman secara cepat bisa hadir saat ini. Bumi akan diguncang dengan kekacauan dan gelombang pasang di udara akan menghasilkan topan yang begitu dahsyat setiap hari. 
Jika kerak bumi memiliki ketebalan 10 kaki, maka tidak akan ada oksigen, sehingga kehidupan binatang tidak akan mungkin; dan jika samudra hanya sedalam 10 kaki, maka karbon dioksida dan oksigen akan diserap dan kehidupan tumbuh-tumbuhan di permukaan tanah tidak akan dimungkinkan. Jika atmosfer lebih tipis, maka beberapa dari meteor yang sekarang terbakar pada bagian luarnya hingga jutaan per harinya akan menimpa seluruh bagian dari permukaan bumi. 
Oksigen biasanya terkandung dalam jumlah 21 persen di atmosfer. Sementara seluruh atmosfer sendiri akan menekan permukaan bumi kira-kira sekitar 15 pound  dalam setiap inci persegi pada ketinggian di atas permukaan laut. Oksigen yang hadir di atmosfer merupakan bagian dari tekanan ini, dengan tekanan sebesar tiga pound tiap inci persegi. Sementara sisa oksigen terkunci dalam bentuk senyawa di kerak bumi dan juga menjadi penyusun dari 8/10 air di seluruh dunia. Oksigen adalah nyawa bagi kehidupan untuk setiap binatang darat dan untuk tujuan ini oksigen tidak bisa diperoleh kecuali melalui atmosfer. 
Pertanyaan muncul bagaimana senyawa kimia yang aktif ini menghindari penggabungan (dengan unsur lain) dan tersisa di atmosfer dalam proporsi yang tepat yang dibutuhkan bagi kehadiran segala makhluk hidup. Jika misalnya, sebagai contoh, alih-alih 21 persen, jumlahnya adalah 50 persen atau lebih di atmosfer, maka setiap substansi kimia yang bisa terbakar di dunia ini akan begitu mudah terbakar hingga pada taraf sebuah sambaran petir yang menghantam pohon akan memicu kebakaran hutan, yang akan memicu ledakan... Jika oksigen bebas, yang mana merupakan satu dari banyak jutaan senyawa kimia di bumi, diserap, maka kehidupan binatang akan musnah. 
Ketika manusia bernapas, dia akan menarik oksigen, yang kemudian diambil oleh darah dan disalurkan ke seluruh tubuh. Oksigen ini akan membakar makanan di setiap sel dengan sangat lambat pada temperatur yang cukup rendah, akan tetapi menghasilkan karbon dioksida dan uap air, sehingga ketika seorang manusia bernapas seperti tungku panasnya, maka ada penjelasan yang menarik di situ. Karbon dioksida kemudian keluar dari paru-parunya dan tidak dapat dihirup kecuali dalam jumlah yang kecil. Kemudian itu akan membuat paru-paru melakukan tindakan berikutnya dan kemudian dalam helaan napas yang berikutnya membuang semua karbon dioksida ke atmosfer. Jadi semua kehidupan binatang  menyerap oksigen dan membuang karbon dioksida. Lebih jauh lagi oksigen sangat penting bagi kehidupan karena pengaruhnya terhadap senyawa lainnya  dalam darah atau pada tempat lainnya di tubuh, di mana tanpanya maka kehidupan akan berhenti. 
Di lain pihak, seperti yang sudah diketahui, semua kehidupan tanaman sangat bergantung pada sebagian kecil dari karbon dioksida yang ada di atmosfer di  mana katakanlah begitu ia bernapas. Untuk menyatakan reaksi fotosintesis yang begitu kompleks dengan cara yang sesederhana mungkin, daun dari pepohonan bertindak sebagai paru-parunya dan memiliki tenaga (daya) ketika terdapat sinar matahari sehingga bisa memisahkan karbon dioksida menjadi karbon dan oksigen. Dengan kata lain, oksigen dilepaskan dan karbon dipertahankan dan digabungkan dengan hidrogen pada air yang diangkut oleh akar dari tanaman. Dengan reaksi kimia yang menakjubkan, dari unsur-unsur ini akan dihasilkan gula, selulosa dan banyak senyawa kimia lainnya yang kemudian membentuk buah,  dan bunga  [semuanya memiliki keharuman yang berbeda, rasa, warna dan bentuk, yang sesuai dengan jenis pohon yang ditanam. Dapatkah perbedaan yang begitu banyak ini disebabkan oleh informasi yang ada pada bibit tanaman, yang buta, bodoh, dan tidak sadar?].  Tanaman memberi makan dirinya sendiri dan kemudian menghasilkan cukup makanan untuk memberi makan semua jenis binatang yang ada di permukaan bumi. Pada saat yang bersamaan,  tanaman juga mengeluarkan oksigen untuk kita hirup yang mana tanpanya hidup akan berakhir dalam lima menit. Jadi semua tanaman, hutan, rumput, lumut, dan semua kehidupan tumbuhan lainnya, membentuk kehidupannya dari karbon dan air. Binatang mengeluarkan karbon dioksida dan karbon tanaman mengeluarkan oksigen. Jika pertukaran ini tidak terjadi, maka kemungkinannya adalah semua kehidupan binatang dan tanaman akan menggunakan seluruh dari oksigen atau seluruh karbon dioksida, sehingga keseimbangan akan menjadi kacau, di mana satunya menjadi layu atau mati sementara yang lainnya akan mengikuti. 
Hidrogen harus dimasukkan, meskipun kita tidak bernapas dengan itu. Tanpa hidrogen maka air tidak akan ada, dan kandungan air pada binatang dan tanaman secara mengejutkan sangat besar, dan benar-benar diperlukan. Oksigen, hidrogen, dan karbon dioksida, dan karbon, secara sendiri-sendiri dan dalam hubungannya dengan unsur lainnya, merupakan unsur paling penting bagi kehidupan biologi. Semuanya adalah yang paling penting pada mana kehidupan bergantung. 
Kita menumpahkan berbagai macam bahan kimia ke dalam laboratorium sistem pencernaan, yang merupakan laboratorium terbesar di dunia yang sama sekali tidak peduli terhadap apa yang kita masuk kan ke dalamnya, di mana bergantung pada apa yang dianggap sebagai proses otomatis yang menjaga kita tetap hidup. Ketika makanan sudah dipecahkan dan kemudian dipersiapkan, mereka kemudian diantarkan terus menerus kepada masing-masing dari miliaran sel-sel yang jumlahnya lebih besar ketimbang jumlah manusia yang ada permukaan bumi. Pengantaran kepada setiap sel-sel haruslah tetap, dan hanya senyawa yang diperlukan oleh tiap-tiap sel untuk merubahnya menjadi tulang, kuku, otot, rambut, mata, dan gigi, akan diambil oleh sel yang sesuai. Di sini laboratorium kimia akan menghasilkan lebih banyak senyawa ketimbang laboratorium apapun yang telah diciptakan oleh kecerdasan manusia. Di sini sistem pengantaran lebih besar dibandingkan metode apapun dalam transportasi atau distribusi yang sudah dikenal oleh dunia, semuanya dilakukan dengan urutan yang sempurna. Dari sejak kecil, hingga mencapai katakanlah lima puluh tahun, laboratorium ini tidak pernah mengalami kesalahan yang berarti, walaupun senyawa yang ditanganinya bisa saja berasal dari jutaan jenis yang berbeda dari molekul dan banyak di antaranya yang mematikan. Ketika saluran pendistribusian tiba-tiba saja menjadi lamban karena penggunaan yang lama maka kita akan merasakan kemampuan yang melemah  yang menandakan usia yang sudah menua. 
Ketika makanan yang tepat diserap oleh tiap sel, maka itu tetap adalah sebatas makanan. Proses yang ada di sel sekarang menjadi bentuk dari pembakaran, yang menjadi sebab bagi panas di seluruh tubuh. Kamu tidak akan bisa melakukan pembakaran tanpa adanya pemantik. Api harus dinyalakan,  dan jadi [kamu dibekali dengannya] sebuah pembakaran kimia yang memicu api yang terkendali bagi oksigen, hidrogen, dan karbon pada makanan di tiap sel, sehingga menghasilkan panas yang dibutuhkan dan, seperti halnya bentuk api lainnya, menghasilkan uap air dan karbon dioksida. Karbon dioksida kemudian diangkut oleh darah ke paru-paru, dan di sana itu adalah suatu hal yang membuat mu menarik napas mu agar tetap hidup. Setiap  orang akan menghasilkan sekitar dua pound karbon dioksida sehari, namun dengan proses yang menakjubkan bisa mengeluarkannya. Setiap binatang akan mencerna makanan, dan setiap makanan tersebut akan memiliki rumus kimia tertentu yang dibutuhkan oleh tiap individu. Bahkan dalam satu menit rincian senyawa kimia pada darah, akan berbeda di setiap spesies. Di sana akan terjadi proses pembentukan khusus untuk setiap spesies itu. 
Dalam kasus infeksi oleh kuman yang berbahaya, sistem juga akan terus mempertahankan pasukannya untuk menghadapinya, dan biasanya mampu mengatasinya dan melindungi struktur keseluruhan dari manusia dari kematian dini. Tidak ada satupun kombinasi yang ajaib seperti itu bisa mengambil tempat dalam lingkungan mana pun tanpa adanya kehidupan. Dan semua ini dilakukan dengan urutan yang sempurna, dan urutan tentu saja berlawanan dengan kebetulan. 
Apakah ini menghendaki dan merujuk pada Sesuatu yang mengetahui segala  urusan kira secara menyeluruh,  beserta lingkungannya dan mekanisme yang ada di tubuh. Sesuatu yang mengetahui segala hal dan melakukan apa yang dikehendakinya. Dalam kata-kata Morrison: tujuan merupakan hal yang paling dasar dalam segala hal, dari hukum yang mengatur seluruh alam semesta sampai pada kombinasi atom yang mempertahankan kehidupan kita. Atom dan molekul pada makhluk hidup akan melakukan hal-hal yang ajaib dan melakukan mekanisme yang menakjubkan, akan tetapi mekanisme tersebut sama sekali tidak berguna, kecuali adanya kecerdasan yang membuat mereka dalam pergerakan dengan tujuan tertentu.  Ada arahan dari kecerdasan yang tidak bisa dijelaskan oleh sains, dan sains tidak bisa mengatakan bahwa itu murni bersifat material.
Mengapa Allah Menciptakan Hukum Alam dan Sebab
Pada kehidupan yang berikutnya, dunia bagi kekuasaan Allah, maka Allah akan melaksanakan kehendaknya secara langsung. Karena tidak ada sebab maka segala hal akan terjadi secara spontan. Akan tetapi di sini, dalam dunia yang penuh kebajikan, maka nama Allah yang  maha bijaksana menghendaki kekuasaannya Allah untuk bertindak dibalik tirai dari sebab dan hukum untuk beberapa alasan, beberapa di antaranya adalah sebagai berikut: 
1. Dua hal yang saling bertentangan selalu hadir bersamaan di dunia: kebenaran dan kepalsuan, cahaya dan kegelapan, baik dan jahat, dan seterusnya. Karena kecenderungan kita sebagai manusia menuju pada baik dan buruk, kita akan diuji untuk melihat apakah kita akan menggunakan kehendak bebas kita atau tidak serta kemampuan lainnya pada jalan yang benar dan baik. Kebijaksanaan Allah menghendaki bahwa sebab-sebab dan hukum-hukum harus menutupi pelaksanaan dari Kehendak Allah. Jika Allah mau, maka bisa saja dia menggerakkan planet-planet dengan tangan-Nya dalam cara yang bisa kita lihat, atau bisa saja mengaturnya dengan perantaraan malaikat-malaikat yang bisa terlihat. Jika ini yang terjadi, maka kita tidak harus membicarakan tentang hukum-hukum dan sebab-sebab yang terlibat. Untuk menyampaikan perintah-perintah-Nya, bisa saja dia berbicara dengan setiap orang secara langsung, tanpa harus mengutus para Nabi. Untuk membuat kita percaya pada keberadaan-Nya dan keesaan-Nya, maka dia bisa saja menulis nama-Nya pada bintang dan langit. Namun kehadiran kita di dunia ini tidak lagi menjadi tempat ujian bagi hamba-hamba-Nya. Sebagai hasil dari ujian ini, sejak kehadiran Nabi Adam dan Hawa, baik dan buruk mengalir di dunia ini ke dunia berikutnya untuk mengisi pengelompokan pada surga dan neraka.
2. Seperti cermin yang terdiri dari dua sisi, keberadaan memiliki dua segi atau dimensi: satu yang nampak dan bersifat kebendaan (material), dunia yang ada dibaliknya, dan (pada banyak kasus) tidak sempurna; dan dimensi spiritual yang sifatnya transparan, murni dan sempurna. Di sana bisa aja ada bahkan memang ada kejadian dan fenomena pada dimensi material yang tidak kita suka. Mereka yang tidak pernah merasakan kebijaksanaan Allah dibalik segala sesuatu bisa saja mengkritik Allah yang  maha suci untuk kejadian dan fenomena tersebut. Untuk mencegah hal tersebut, Allah membuat hukum-hukum dan sebab-sebab di alam sebagai tirai dari tindakan-Nya.  Sebagai contoh agar kita tidak akan mengkritik Allah atau malaikat pencabut nyawa menjadi sebab bagi kematian kita atau orang lain,   Allah menciptakan penyakit atau bencana alam (di antara sebab-sebab lainnya) antara diri-Nya dan kematian.
Karena ketidaksempurnaan yang hakiki pada dunia ini, maka kita akan merasakan banyak kekurangan dan kesusahan.  Menjadi sebuah kemutlakan, bahwa setiap kejadian dan fenomena adalah baik pada hakikatnya dan pada akibat yang ditimbulkan. Kapanpun Allah bertindak dan menetapkan sesuatu adalah pasti baik, indah, dan adil. Ketidakadilan, kejelekan,  dan kejahatan adalah nampak dari luarnya dan hanya sepintas, dan berangkat dari kesalahan dan penyelewengan manusia. Sebagai contoh, seorang hakim bisa saja memutuskan perkara kita secara tidak adil, tapi kita harus tahu bahwa takdir membolehkan hal tersebut karena adanya kejahatan yang tersembunyi yang pernah kita lakukan. Apapun yang menimpa kita adalah akibat kesalahan yang kejahatan yang pernah kita lakukan.  Akan tetapi, mereka yang tidak mengerti akan kebijaksanaan Allah dibalik setiap kejadian dan fenomena akan menghubungkan apa yang nampak sebagai kejelekan dan kejahatan, ketidaksempurnaan dan cela, secara langsung kepada Allah, meskipun dia tidak memiliki cacat atau ketidaksempurnaan. 
Untuk mencegah kesalahan tersebut, keagungan dan kebesaran-Nya menghendaki bahwa sebab-sebab dan hukum-hukum di alam untuk menutupi tindakan-Nya, sementara keimanan pada keesaan-Nya menginginkan bahwa setiap sebab-sebab dan hukum-hukum tersebut tidak bisa dihubungkan kepada segala sesuatu selain Allah. 
1. Jika Allah yang maha suci bertindak di sini secara langsung, maka kita tidak akan mungkin mengembangkan sains, bisa mengetahui kebahagiaan, dan bebas dari ketakutan dan kegelisahan. Segala puji bagi Allah yang bekerja di balik sebab-sebab dan hukum-hukum di alam, maka kita dapat mengamati dan mempelajari pola dari setiap fenomena. Selain itu, setiap kejadian adalah sebuah keajaiban. Aliran dan perubahan dari kejadian dan fenomena membuatnya bisa dipahami oleh kita, dan membangkitkan keinginan kita untuk bertanya dan bercermin darinya, yang merupakan faktor dasar dari sains. Untuk alasan yang sama, hingga derajat tertentu kita bisa merencanakan dan mengatur segala urusan kita. Apa jadinya hidup kita jika kita sama sekali tidak bisa memastikan apakah besok matahari akan bersinar atau tidak?
2. Siapapun yang memiliki sifat-sifat seperti keindahan dan kesempurnaan  menginginkan untuk mengetahuinya  dan membuatnya untuk diketahui. Allah memiliki sifat-sifat  keindahan dan kesempurnaan  dan tidak bergantung pada apapun dan tidak menginginkan apapun. Dia memiliki cinta yang suci dan melebihi batas apapun, dan memiliki keinginan yang suci dalam mewujudkan keindahan-Nya dan kesempurnaan-Nya. Jika dia mewujudkan nama dan sifat-sifat-Nya secara langsung, maka kita tidak akan bisa bertahan darinya. Dia kemudian mewujudkannya dibalik sebab-sebab dan hukum-hukum, dalam derajat yang dibatasi oleh ruang dan waktu, sehingga kita dapat membangun hubungan dengannya, dan bercermin padanya dan merasakannya. Perwujudan secara bertahap dari nama-nama dan sifat-sifat Allah adalah alasan bagi keingintahuan kita serta mengapa kita bertanya-tanya tentangnya.
Empat poin ini merupakan penyusun bagi beberapa alasan mengapa Allah bertindak melalui hukum-hukum dan sebab-sebab di alam. 


\end{document}
